    \frame{\sectionpage}

    \begin{frame}{Symplectic reduction of wavefunctions}
    Study deformation quantisation of \(X\),
    
    Consider the intersection and look for a wavefunction associated to \(\mathcal{O}_{G \cap {\mathbb{L}}} \).
    
    Want to check if this agrees with the formula for the wavefunction:
    \begin{thm}[Reduction of wavefunctions]
    \[ \psi_{\mathcal{B}} = \int \psi_G \, \psi_{\mathbb{L}} \]
    \end{thm}
    \end{frame}
    
    \begin{frame}{Goal}
        Verify the two possible wavefunctions agree:
    \end{frame}
    
    \begin{frame}{An example}
    \begin{ex}    
    \[ \mathcal{O}(W) =   \mathbf{k}[x_1,x_2,y_1,y_2].\] 
    Poisson bracket: 
    \(\{x_i,x_j\}=\{y_i,y_j\}=0, \; \{x_i,y_i\}=1 \). 
    Linear Lagrangian defined by the ideal \(I(\mathbb{L}) = \langle H_1, H_2\rangle\), where:
    \begin{align*}
        H_1 = y_1 + A x_1 +  B x_2,    \\
        H_2 = y_2 +  C x_1 + D x_2. 
    \end{align*}
    Need \(B=C\) for coisotropic.
    
    Let \(G\) be determined by the equation:
    \[ I(G) = \langle H_G =a x_1 + b x_2 + c y_1 + d y_2 \rangle.\]
    \end{ex}
    \end{frame} 

    \begin{frame}
    Transversality is given by requiring at least one of following terms to be non zero:
    \begin{align}
    \label{eqn:transcons2dmin}
    a - cA - dB, \\ 
    b - cB-dD.
    \end{align}
    \end{frame}
    
    \begin{frame}
    
    \(G\) is codimension \(n-g=1\), the extension \(X = W\oplus G/G^{\perp}\) necessarily must have dimension \(6\). 
    Let \[ \mathcal{O}(X) = \mathbf{k}[x_1,x_2,z_1, y_1,y_2,w_1].\] 
    Poisson bracket on \(\mathcal{O}(X)\) 
    \[ \{x_i,y_i\}=\{z_1,w_1\} = 1\]
    Fibre product represented by a map
    \begin{equation} 
    \label{eqn:extgmaprings}
    \mathbf{k}[x_1,x_2,z_1,y_1,y_2,w_1] \rightarrow \mathbf{k}[x_1,x_2,y_1,y_2],
    \end{equation}
    such that the preimage of \(G\) in \(X\), is a linear Lagrangian. 
    \end{frame}
    
    
    \begin{frame}
    
    There is an ideal \(I(G_{Ext})\) in \(\mathcal{O}(X)\) involutive under the Poisson bracket. The map (\ref{eqn:extgmaprings}) is constructed on the generators. First 
    \( x_i  \rightarrow x_i ,
    y_i  \rightarrow y_i\).
    
    Then finding \(G_{Ext}\) as a Lagrangian in \(X\), requires finding two additional linear functions \(\zeta\), and \(\xi\):
    \begin{align}
    \label{eqn:solveforext}
    z_1 &= \zeta_1(x_{\sbt},y_{\sbt}), \\
    w_1 &= \xi_1(x_{\sbt},y_{\sbt}),
    \end{align}
    One choice is 
    \[\zeta_1 = a x_1 + c y_1, \quad  \text{and} \quad  \xi_1 = \frac{1}{cd} \left( d x_1 - c x_2 \right) .\]
    
    \end{frame}
    
    
    \begin{frame} \(\psi_G(x_1,x_2,z_1)\) (associated with \(G_{Ext}\) satisfies the equations: 
    \begin{align*}
       (a x_1 + b x_2) \psi_G + \hslash \left( c \frac{\partial}{\partial x_1 } +  d \frac{\partial}{\partial x_2} \right) \psi_G & = 0, \\
       ( z_1 - a x_1 ) \psi_G  - c \hslash \frac{\partial}{\partial x_1} \psi_G &= 0, \\ 
       \hslash \frac{\partial}{\partial z_1} \psi_G - \frac{1}{c d} ( d x_1 - c x_2) \psi_G &= 0.
    \end{align*}
    
    \end{frame}
    
    
    \begin{frame}
    Looking for a solution of the form:
    \[ \psi_G = C \,\exp\left( -\frac{1}{2} x \cdot K \cdot x  + J  \cdot x\right),\]
    Gives
    \begin{align*}
        K = \frac{1}{\hslash} \left(\begin{array}{cc}
            a/c & 0 \\
            0 & b/d
        \end{array}\right), \quad J = \frac{1}{\hslash} \,  z_1 \left( \begin{array}{c}
            1/c \\
            -1/d
        \end{array}\right).
    \end{align*}
    Similar solution for \( \psi_{\mathbb{L}}\):
    \begin{align*}
        \psi_{\mathbb{L}} =  C' \, \exp \left( -\frac{1}{2}   x \cdot Q \cdot  x \right),
    \end{align*}
    where 
    \[ Q = - \frac{1}{\hslash} \left( \begin{array}{cc}
        A  &  B \\
        B & D  
    \end{array}\right),\]
    \end{frame}
    
    \begin{frame}{Finally perform the integral}
    Fourier like integral 
         \begin{align*} 
         \psi_{\mathcal{B}} &= \int Dx \, \psi_{\mathbb{L}} \psi_{G}  \\
         &= \text{const} \, \exp \left( \frac12\, J\cdot (K+Q)^{-1} \cdot J \right) \\
         & = \text{const}\, \exp\left( -\frac{1}{2 \hslash}\, c_0\, z_1^2\right),
    \end{align*}
    \[ c_0 = \frac{\left(a c-A c^2+d (b-2 B c-d D)\right)}{ c d  \left(a (b-d D)-c \left(A (b-d D)+B^2 d\right)\right)}.\]
    Integral fails precisely when \(G\) and \( \mathbb{L}\) intersect non-transversally.
    \end{frame}
    
    \begin{frame}{The check}
    Take equations for \(G\), \(\mathbb{L}\) in \(X\) and eliminate \(x_{\sbt}\) and \( y_{\sbt}\):
    \begin{equation}
        \label{eqn:check1}
        w_1 = c_1 z_1,
    \end{equation}
    Algebraically, represents the sum of the ideals of \(G\) and \( \mathbb{L}\), per fibre product. Note that \[c_1 = -c_0.\]  
    Can quantise, look for dual module, find equation:
    \[ \hslash \frac{\partial}{\partial z_1} \psi = c_1 z_1 \psi. \]
    This equation has a solution of the form:
    \[ \psi =  \mathrm{const} \, \exp\left( \frac{1}{2 \hslash} c_1 z_1^2 \right) = \text{const}\, \exp\left( -\frac{1}{2 \hslash}\, c_0\, z_1^2\right).\]
    
    \end{frame}
    
    \begin{frame}
        This verifies the integral formula in a simple case. 
    \end{frame}