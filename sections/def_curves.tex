\frame{\sectionpage}

\begin{frame}{Setup}
Apply ideas to 
\[V_\Sigma =\{ \eta\in H^0(\Omega^1(\Sigma-R)) \mid 0=\oint_{a_i}\eta\}\]
Define 
\begin{align*} 
W_\Sigma &= V_\Sigma \oplus V_\Sigma^* \\ 
G_\Sigma &= \{ \eta\in H^0(\Omega^1(\Sigma-R)) \mid\Res_{r\in R}\eta=0\} \\
\mathcal{H}_\Sigma &= W_\Sigma /\!\!/G_\Sigma^{\perp} = H^1(\Sigma, \mathbb{C}) \cong \mathbb{C}^{2g}\\
\overline{V}_\Sigma &= V_\Sigma  /\!\!/G_\Sigma^{\perp} \subset  H^1(\Sigma, \mathbb{C})
\end{align*}

\end{frame}


\begin{frame}{Defines an Airy structure}
\(\Sigma\) surface, \(R\) divisor.
\[ A_\Sigma = {a_{\Sigma}}_{ijk} x^i x^j x^k  \cong \frac{1}{4} \sum_{\alpha \in R} x^{1,\alpha} \otimes x^{1,\alpha} \otimes x^{1,\alpha}   \]

Want to study image of \(A_\Sigma\) under symplectic reduction.
\end{frame}



\begin{frame}{Period matrix}
Coordinates \((z^1,\dots z^g, w_1, \dots w_g )\) on \(\mathcal{H}_\Sigma\).
\vspace{1em}
Define the  matrix of a Riemann surface \( \tau_{ij}(z_1, \dots, z_g)\).

\end{frame}

\begin{frame}{A tensor} 

Variation of the period matrix is equivalent to the reduction of \( A_\Sigma\). Gives the data of the Donagi markman cubic.
\begin{thm}
Under the map
\(V_\Sigma \otimes V_\Sigma \otimes V_\Sigma \rightarrow \overline{V}_\Sigma \otimes \overline{V}_\Sigma \otimes \overline{V}_\Sigma\),
\( A_\Sigma \rightarrow \overline{A}_\Sigma\), and furthermore
\[ \frac{\partial \tau_{ij}}{\partial z_k} = \overline{{a_\Sigma}}_{ijk}\]
\end{thm}

Generalises work by Baraglia and Huang.
\end{frame}


